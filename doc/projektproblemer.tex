\documentclass[a4paper, oneside]{memoir}
\usepackage[danish]{babel} % load typographical rules for the english language
\usepackage{graphics} % for \scalebox
\usepackage{hyperref} % for \href
\usepackage{xcolor} % for text color
\usepackage{enumitem} % for ordered and unordered list
\usepackage{graphicx} % for images
\usepackage{pdfpages} % for including pdfs
\usepackage{footnote} % for footnotes
\usepackage{longtable} % for tabular environment that spans multiple pages and supports footnotes
\usepackage{colortbl} % for cell coloring
\usepackage{multirow} % for \multicolumn

% https://github.com/latex3/babel/issues/51
\makeatletter\AtBeginDocument{\let\@elt\relax}\makeatother

% styling
\setsecnumdepth{subsubsection} % how deep to number sections
\setlength{\parindent}{0em} % horizontal indent for first line of paragraph
\setlength{\parskip}{1em} % vertical space between paragraphs

\newcommand{\textdesc}[1]{\textit{\textbf{#1}}}
\newcommand{\descitem}[1]{\item \textdesc{#1}}

\title{\documenttitle\\\scalebox{0.85}{\documentsubtitle}}
\author{Aslak Johansen \href{mailto:asjo@mmmi.sdu.dk}{asjo@mmmi.sdu.dk}\\Aisha Umair \href{mailto:aiu@mmmi.sdu.dk}{aiu@mmmi.sdu.dk}}

\begin{document}

\maketitle
\setcounter{tocdepth}{2}
\tableofcontentswrapper

\chapter{Projektproblemer: flere eksempler}
I Tabel 1 er vist nogle gode og dårlige eksempler på projektproblemer. Selvom du ikke kender den nærmere sammenhæng, så prøv alligevel at besvare nedenstående spørgsmål:
\begin{enumerate}
    \item Hvad kunne være nogle grunde til at formuleringerne til venstre er udpeget som dårlige eksempler. mens formuleringerne til højre er udpeget som gode eksempler?
    \item Hvad kunne der stå i stedet for spørgsmålstegnet?
    
\end{enumerate}

\begin{center}
  \begin{tabular}{|p{5cm} | p{5cm}|}
    \hline
  \multicolumn{2}{c}{\textbf{Eksempler på problemer i et projekt}} 
 \\
    \hline
  \textbf{Dårligt eksempel} & \textbf{Bedre eksempel}
 \\
 Vi sender for lidt papir til genbrug & Virksomhed XXX’s stigende forbrug af papir er i modstrid med virksomhedens tilslutning til FN’s verdensmål, specielt #15 Livet på land.  
    \\
    \hline
    Der mangler motorvejsspor mellem Odense og Middelfart & Borgerne bruger for lang tid på at pendle mellem Odense og Middelfart
     \\
    \hline
    Der mangler et system, så der bliver grøn bølge for udrykningskøretøjerne & ?
     \\
    \hline
    \end{tabular}
    \\
\caption{Table 1: Eksempler på problemer i et projekt (opgave)}
     \label{tab:table1}
    \end{center}
    I Tabel 2 er vist nogle eksempler på åbne og lukkede problemformuleringer. Hvad kunne der stå i stedet for spørgsmålstegnene?
    
    \begin{center}
  \begin{tabular}{|p{4cm}| p{5cm}|p {5cm}|}
    \hline
  \textbf{Centralt problem} & \textbf{Eksempel på åben problemformulering} & \textbf{Eksempel på problemformulering der inddrager en løsningsmulighed}
 \\
    \hline
  Virksomhed XXX’s stigende forbrug af papir i administrationen er i modstrid med virksomhedens tilslutning til FN’s verdensmål, specielt #15 Livet på land. & Hvordan kan virksomhed XXX styrke virksomhedens tilslutning til FN’s verdensmål ved at reducere forbruget af papir?  & Hvordan kan virksomhed XXX styrke virksomhedens tilslutning til FN’s verdensmål ved at reducere forbruget af papir uden forringelse af den interne kommudnikation ved at styrke genbrug og anskaffelse af digitalt udstyr der kan erstatte papir.
 \\
    \hline
 Borgerne bruger for lang tid på at pendle mellem Odense og Middelfart & Kan vi mindske pendlingstiden mellem odense og Middelfart? & Kan vi mindske pendlingstiden ved at indføre en mere fleksibel kollektiv trafik?
    \\
    \hline
    (Dit svar fra opgave 2.2) & ? & ?
     \\
    \hline
    \end{tabular}
    \caption{Table 2: Eksempler på problemformuleringer (opgave)}
     \label{tab:table2}
    \end{center}
    

\end{document}