\documentclass[a4paper, article, oneside]{memoir}
\counterwithout{section}{chapter}
\usepackage[english]{babel} % load typographical rules for the english language
\usepackage{graphics} % for \scalebox
\usepackage{hyperref} % for \href
\usepackage{xcolor} % for text color
\usepackage{enumitem} % for ordered and unordered list
\usepackage{graphicx} % for images
\usepackage{pdfpages} % for including pdfs

% https://github.com/latex3/babel/issues/51
\makeatletter\AtBeginDocument{\let\@elt\relax}\makeatother

% styling
\setsecnumdepth{subsubsection} % how deep to number sections
\setlength{\parindent}{0em} % horizontal indent for first line of paragraph
\setlength{\parskip}{1em} % vertical space between paragraphs

\newcommand{\textdesc}[1]{\textit{\textbf{#1}}}
\newcommand{\descitem}[1]{\item \textdesc{#1}}

\title{Project Description}
\author{Aslak Johansen}

\begin{document}

\maketitle
\setcounter{tocdepth}{2}
\tableofcontents


\section{Iteration \#1}

Ved afslutningen af iteration \#1 afleveres C\#-projektet (se afsnit \ref{sec:code} herunder) i sin foreløbige form og et udkast til projektporteføljen (se afsnit \ref{sec:report} herunder).
 
Udkastet kan omfatte større eller mindre dele af den endelige projektportefølje efter gruppens eget valg og aftale med vejleder. Det er vigtigt, at udkastet præsenterer resultaterne af både problemanalysefasen og første iteration, dvs. at afsnittene 1-7  er påbegyndt. Det er ikke vigtigt, at al teksten er fuldt ud gennemarbejdet. Hvor gennemarbejdet teksten skal være afhænger af, hvor langt I er i arbejdsprocessen, og hvad I gerne vil have feedback på i forbindelse med afleveringen. Tekst med lav færdiggørelsesgrad kan bruges til at få feedback på, om I er på sporet i projektet, jeres måde at arbejde på, jeres prioritering af stoffet mm. Tekst med høj færdiggørelsesgrad kan bruges til at få nærmere feedback på, hvordan I scorer i forhold til målene for semesterprojektet.

\section{Endelig Aflevering}

Ved afslutningen af projektet afleveres det endelige C\#-projekt og den endelige projektportefølje. C\#-projektet og projektporteføljen bruges som grundlag for jeres forberedelse til den mundtlige projekteksamen. 

\section{C\# Projekt}
\label{sec:code}

C\#-projektet afleveres som en zip-fil. Øvrige krav til C\#-projektet fremgår af projektbeskrivelsen (se på itslearning under Planer/Semesterprojekt og Projektbeskrivelse).

\newpage
\section{Projektportefølje}
\label{sec:report}

Den endelige projektportefølje skal have følgende opbygning:
\begin{enumerate}
  \item[i] Forside
  \item[ii] Indholdsfortegnelse

  \item[1] Problemanalyse
  \item[2] Krav
  \item[3] Design
  \item[4] Implementering
  \item[5] Evaluering
  \item[6] Konklusion
  \item[7] Procesevaluering

  \item[Bilag A] Samarbejdsaftale
  \item[Bilag B] Vejlederaftale
  \item[Bilag C] Projektlog
  \item[Bilag D] Kontrolskema for portefølje
\end{enumerate}

Projektporteføljen skal overholde retningslinjer og krav i “Kontrolskema for projektportefølje” (se afsnit \ref{sec:control}). Prohjektportefølgen må ikke overstige 40 sider (uden bilag). Den skal dække ovenstående indhold fyldestgørende. Det vil nok kræve omkring 35 sider.

Porteføljen skal følge retningslinjerne i:
\begin{itemize}
  \item Dahl mfl (2016): Kap. 8 uddrag vedr. evaluering\footnote{Tilgængelig på itslearning under fase 4.}
  \item Larsen (2021): Kapitel 13 Professionel og klar formidling … (side 249-260)\footnote{\url{https://projekterograpporterpaatekniskeudd-2udg.digi.hansreitzel.dk/?id=157}}
\end{itemize}

\section{Kontrolskema for Projektportefølje}
\label{sec:control}

Skemaet udfyldes ved i kolonnen ”Opfyldt” at markere de krav, der menes at være opfyldt med et + og de uopfyldte med. Skemaet indsættes som bilag D.

\newpage
\begin{center}
\begin{longtable}{|l|p{.6\textwidth}|c|}
  \hline
    \textbf{Kapitel} & \textbf{Indhold} & \textbf{Opfyldt +/-} \\
  \hline
    \textbf{Forside}
    &
    Projekttitel, uddannelsesinstitution, fakultet, institut, uddannelse, semester, kursuskode, projektperiode, vejleder, projektgruppe og projektdeltagere (fornavn, efternavn, sdu-email). Må gerne have illustrationer.
    &
    \\
  \hline
    \textbf{Indholdsfortegnelse}
    &
    Samlet indholdsfortegnelse for hele projektporteføljen.
    &
    \\
  \hline
    \textbf{Indledning}
    &
    Projektets rammer og baggrunden for projektet.
    \par
    Problemanalysen.
    Redegørelse for det udleverede framework.
    \par
    Problemformulering og afgrænsninger.
    \par
    Metoder.
    \par
    Tidsplanen.
    &
    \\
  \hline
    \textbf{Krav}
    &
    De samlede krav (modtagne og formuleret af jer). Analyse af kravene, herunder:
    \par
    $\bullet$ Noun-Verb analyse
    \par
    $\bullet$ CRC kort
    &
    \\
  \hline
    \textbf{Design}
    &
    Beskrivelsen af designet, herunder et eller flere UML-klassediagrammer. I beskrivelsen indgår beskrivelse af arkitektur og brugergrænsefladedesign.
    &
    \\
  \hline
    \textbf{Implementering}
    &
    Beskrivelse af implementering med centrale dele af programkoden.
    &
    \\
  \hline
    \textbf{Evaluering}
    &
    Evaluering af om løsningen lever op til kravene, fx gennem en brugerundersøgelse eller brugerafprøvning. Beskrivelse af hvad der er opnået og hvad der ikke er opnået i projektet i forhold til det -- som i indledningen beskrevet -- forventede. Beskrivelse af styrker og svagheder ved resultaterne og om der kunne være opnået bedre resultater.  Beskrivelse af kendte fejl.
    &
    \\
  \hline
    \textbf{Iteration \#1 og \#2}
    &
    Er der i ovenstående afsnit (Krav, design, Implementering og Evaluering) medtaget resultater både fra iteration \#1 og fra iteration \#2?
    &
    \\
  \hline
    \textbf{Konklusion}
    &
    Opsummering af resultaterne og evalueringen. Svar på problemformuleringen. 
    &
    \\
  \hline
    \textbf{Procesevaluering}
    &
    Processen og gruppens refleksion over processen:
    \par
    Læringsprocessen, teamroller, samarbejdet internt i gruppen og med vejleder, projektarbejdsformen, arbejdsformer, metoder, skriveprocessen, den tidsmæssige styring af projektet,ledelse af projektet, arbejdsfordeling i projektet m.m.
    &
    \\
  \hline
    \textbf{Samarbejdsaftale}
    &
    Samarbejdsaftalen indsættes
    &
    \\
  \hline
    \textbf{Vejlederaftale}
    &
    Vejlederaftale indsættes
    &
    \\
  \hline
    \textbf{Projektlog}
    &
    Adresse på og et link til projektloggen
    &
    \\
  \hline
    \textbf{Kontrolskema}
    &
    Udfyldt Kontrolskema
    &
    \\
  \hline
\end{longtable}
\vspace{-1cm} % to keep latex from adding a blank page
\end{center}
\end{document}
