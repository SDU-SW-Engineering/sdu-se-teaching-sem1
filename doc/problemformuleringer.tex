\documentclass[a4paper, oneside]{memoir}
\usepackage[danish]{babel} % load typographical rules for the english language
\usepackage{graphics} % for \scalebox
\usepackage{hyperref} % for \href
\usepackage{xcolor} % for text color
\usepackage{enumitem} % for ordered and unordered list
\usepackage{graphicx} % for images
\usepackage{pdfpages} % for including pdfs
\usepackage{footnote} % for footnotes
\usepackage{longtable} % for tabular environment that spans multiple pages and supports footnotes
\usepackage{colortbl} % for cell coloring
\usepackage{multirow} % for \multicolumn

% https://github.com/latex3/babel/issues/51
\makeatletter\AtBeginDocument{\let\@elt\relax}\makeatother

% styling
\setsecnumdepth{subsubsection} % how deep to number sections
\setlength{\parindent}{0em} % horizontal indent for first line of paragraph
\setlength{\parskip}{1em} % vertical space between paragraphs

\newcommand{\textdesc}[1]{\textit{\textbf{#1}}}
\newcommand{\descitem}[1]{\item \textdesc{#1}}

\title{\documenttitle\\\scalebox{0.85}{\documentsubtitle}}
\author{Aslak Johansen \href{mailto:asjo@mmmi.sdu.dk}{asjo@mmmi.sdu.dk}\\Aisha Umair \href{mailto:aiu@mmmi.sdu.dk}{aiu@mmmi.sdu.dk}}

\begin{document}

\maketitle
\setcounter{tocdepth}{2}
\tableofcontentswrapper

\begin{savenotes}
\chapter{Problemformuleringer}
Nedenstående eksempler har fokus på verdensmål #13 om klimaindsats
%%%%%%%%%%%%%%%%%%%%%%%%%%%%%%%%%%%%%%%%%%%%%%%%%%%%%%%%%%%%%%%%%%%%%%%%%%%%%%%%
%%%%%%%%%%%%%%%%%%%%%%%%%%%%%%%%%%%%%%%%%%%%%%%%%%%%%%%%%%%%%%%%%%%%%%%%%%% Ulovlig træfældning.


\section{Ulovlig træfældning.}
Når træ fældes ulovligt, så kan det have omfattende negative økonomiske, miljømæssige og samfundsmæssige virkninger. Ulovlig skovhugst medfører tabte indtægter, underminerer lovlig virksomhed og kan have alvorlige miljømæssige konsekvenser som tab af biodiversitet og udledning af drivhusgasser med påvirkning af klimaet.
\\ CLIM er en NGO (en interesseorganisation) som bl.a. har til formål at formidle viden om træhugst og formidler i dag viden via deres hjemmeside. CLIM ønsker at få en mere virksom formidling af viden om træhugst. ulovlig fældning af træ og konsekvenserne af det for at øge opmærksomheden på de handlemuligheder, som det enkelte menneske har for at påvirke udviklingen, fx gennem køb af certificeret træ\footnote{CLIM er en opdigtet NGO}.
\end{savenotes}

%%%%%%%%%%%%%%%%%%%%%%%%%%%%%%%%%%%%%%%%%%%%%%%%%%%%%%%%%%%%%%%%%%%%%%%%%%%%%%%%
%%%%%%%%%%%%%%%%%%%%%%%%%%%%%%%%%%%%%%%%%%%%%%%%%%%%%%%%%%%%%%%%%%%%%%%%%%% Eksempel 1 - Formidling af vide

\section{Eksempel 1 - Formidling af vide}
I dette eksempel handler problemet om at forbedre noget eksisterende (formidling af information).
\begin{center}
  \begin{tabular}{|p{3cm}|p{11cm}|}
    \hline
   Problem: &
  Ineffektiv formidling af information om de handlemuligheder det enkelte menneske har for at påvirke udviklingen af ulovlig træhugst
 \\
    \hline
  Problemformulering: & Hvordan kan CLIM forbedre formidlingen af information om de handlemuligheder det enkelte menneske har for at påvirke udviklingen af ulovlig træhugst?
 \\
    \hline
     Underspørgsmål: & 
     \begin{tabular}{p{11cm}}
     Hvad er CLIM? 
      \\ Hvordan formidler CLIM viden om træhugst, konsekvenser og handlemuligheder i dag? 
      \\ Hvad er træhugst?
      \\ Hvad er lovlig og ulovlig træhugst? 
      \\ Hvad er konsekvenserne af ulovlig træhugst? 
      \\ Hvilke handlemuligheder har det enkelte menneske 
      \\ Hvordan kan information om handlemuligheder formidles mere effektivt?
     \end{tabular}
 \\
    \hline
    \end{tabular}
    \end{center}

%%%%%%%%%%%%%%%%%%%%%%%%%%%%%%%%%%%%%%%%%%%%%%%%%%%%%%%%%%%%%%%%%%%%%%%%%%%%%%%%
%%%%%%%%%%%%%%%%%%%%%%%%%%%%%%%%%%%%%%%%%%%%%%%%%%%%%%%%%%%%%%%%%%%%%%%%%%% Eksempel 2 - LumberJacket - et læringsspil

\section{Eksempel 2 - LumberJacket - et læringsspil}
I dette eksempel anskues problemet som et behov for at designe noget nyt fra bunden (læringsspil).
\begin{savenotes}
\begin{center}
  \begin{tabular}{|p{3cm}|p{11cm}|}
    \hline
   Problem &
  Udvikling af et læringsspil, der giver viden om de handlemuligheder det enkelte menneske har for at påvirke udviklingen af ulovlig træhugst
 \\
    \hline
  Problemformulering & Hvordan kan CLIM forbedre viden om de handlemuligheder det enkelte menneske har for at påvirke udviklingen af ulovlig træhugst gennem udvikling af et læringsspil?
 \\
    \hline
     Underspørgsmål: & 
    \begin{tabular}{p{11cm}}
     Hvad er CLIM? 
      \\ Hvordan formidler CLIM viden om træhugst, konsekvenser og handlemuligheder? 
      \\ Hvad er træhugst?
      \\ Hvad er lovlig og ulovlig træhugst? 
      \\ Hvad er konsekvenserne af ulovlig træhugst? 
      \\ Hvilke handlemuligheder har det enkelte menneske? 
      \\ Hvad er et læringsspil? 
      \\ Hvordan kan vi forbedre viden om træhugst, konsekvenser og handlemuligheder i et læringsspil? \\ Hvordan udvikles et læringsspil?
     \end{tabular}
 \\
    \hline
    \end{tabular}
    \end{center}
    
    %%%%%%%%%%%%%%%%%%%%%%%%%%%%%%%%%%%%%%%%%%%%%%%%%%%%%%%%%%%%%%%%%%%%%%%%%%%%%%%%
%%%%%%%%%%%%%%%%%%%%%%%%%%%%%%%%%%%%%%%%%%%%%%%%%%%%%%%%%%%%%%%%%%%%%%%%%%% Input til problemformuleringerne

\section{Input til problemformuleringerne}
Problemformuleringerne herover bygger på nedenstående arbejde. Tabel 1 viser komponenter der kan indgå i hovedspørgsmålet i en problemformulering:
\begin{center}
  \begin{tabular}{|p{3.2cm} p{7.5cm}|p {2.5cm}|}
    \hline
  \textbf{Komponenter} & & \textbf{Uddybning}
 \\
    \hline
  \textbf{Spørgeord} & Hvordan kan ...  & Medtages altid
 \\
    \hline
 \textbf{Aktør} & Den organisation, afdeling eller enhed, der ønsker problemet afhjulpet. &   
    \\
    \hline
    \textbf{Handling} & Et af de to verber at øge eller at reducere eller synonymer & Medtages altid
     \\
    \hline
    \textbf{Variabel} & Den variabel, der ønskes forbedret. & Medtages altid
     \\
    \hline
    \textbf{Specificeringer} & Fortæller hvor og hvornår problemet findes &
     \\
    \hline
    \textbf{Forudsætninger} & Fortæller hvilke andre variable der ikke må forringes. &
     \\
    \hline
    \textbf{Metodeafgrænsning} & Afgrænser projektet til at anvende en bestemt metode (fx nydesign, redesign eller planlægning). & Medtages altid i projekter der designer noget nyt fra bunden.
    \\
    \hline
    \end{tabular}
     \caption{Table 1: Komponenter i hovedspørgsmål (Larsen, 2018, side 43)}
     \label{tab:table1}
    \end{center}
    I Tabel 2 vises to minimale eksempler på hovedspørgsmål fra henholdsvist et projekt, der forbedrer noget eksisterende og et projekt, der designer noget nyt fra bunden:
    
    \begin{center}
  \begin{tabular}{|p{3.2cm}| p{5.5cm}|p {4.5cm}|}
    \hline
  \textbf{Komponenter} & \textbf{Projekt der forbedrer noget eksisterende} & \textbf{Projekt der designer noget nyt fra bunden}
 \\
    \hline
  \textbf{Spørgeord} & Hvordan kan vi ...  & Hvordan kan vi
 \\
    \hline
 \textbf{Handling} & forbedre \footnote{Her bruges ordet “Forbedre”, fordi det at forbedre formidling af information og viden ses som en multivariat størrelse, hvor det skal fastlægges, hvad der nærmere skal øges og reduceres.}. & øge
    \\
    \hline
    \textbf{Variabel} & formidlingen af information om de handlemuligheder det enkelte menneske har for at påvirke udviklingen af ulovlig træhugst & viden om de handlemuligheder det enkelte menneske har for at påvirke udviklingen af ulovlig træhugst
     \\
    \hline
    \textbf{Metodeafgrænsnin} & & gennem udvikling af et læringsspil
     \\
    \hline
    \end{tabular}
    \caption{Table 2: Eksempler på hovedspørgsmål, der bruger komponenterne i Tabel 1\footnote{(Larsen, 2018, 41-47) skelner mellem problemformuleringer for projekter der handler om en situation hvor noget eksisterende kan forbedres og projekter, hvor der er et behov for at designe noget nyt fra bunden. Det gør vi ikke.}}
     \label{tab:table2}
    \end{center}
\end{savenotes}    
    %%%%%%%%%%%%%%%%%%%%%%%%%%%%%%%%%%%%%%%%%%%%%%%%%%%%%%%%%%%%%%%%%%%%%%%%%%%%%%%%
%%%%%%%%%%%%%%%%%%%%%%%%%%%%%%%%%%%%%%%%%%%%%%%%%%%%%%%%%%%%%%%%%%%%%%%%%%% Underspørgsmål
\newpage
\section{Underspørgsmål}
Underspørgsmålene udfolder hovedspørgsmål . Det handler om at stille de underspørgsmål, der er nødvendige for at besvare hovedspørgsmålet. Gode underspørgsmål danner grundlag for metodevalg og for planlægning. Gode underspørgsmål peger på behov for analyse (“Hvad”) og design (“Hvordan”). Eksempler på underspørgsmål:
\begin{itemize}
    \item Hvad er CLIM?
    \item Hvordan formidler CLIM viden om træhugst, konsekvenser og handlemuligheder?
    \item Hvad er træhugst?
    \item Hvad er lovlig og ulovlig træhugst?
    \item Hvad er konsekvenserne af ulovlig træhugst?
    \item Hvilke handlemuligheder har det enkelte menneske?
    \item Hvad er et læringsspil?
    \item Hvordan kan vi forbedre viden om træhugst, konsekvenser og handlemuligheder i et læringsspil?
    \item Hvordan udvikles et læringsspil?
\end{itemize}

\end{document}