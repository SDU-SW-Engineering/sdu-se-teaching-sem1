\documentclass[a4paper,article, oneside]{memoir}
\usepackage[danish]{babel} % load typographical rules for the english language
\usepackage{graphics} % for \scalebox
\usepackage{hyperref} % for \href
\usepackage{xcolor} % for text color
\usepackage{enumitem} % for ordered and unordered list
\usepackage{graphicx} % for images
\usepackage{pdfpages} % for including pdfs
\usepackage{footnote} % for footnotes
\usepackage{longtable} % for tabular environment that spans multiple pages and supports footnotes
\usepackage{colortbl} % for cell coloring
\usepackage{multirow} % for \multicolumn

% https://github.com/latex3/babel/issues/51
\makeatletter\AtBeginDocument{\let\@elt\relax}\makeatother

% styling
\setsecnumdepth{subsubsection} % how deep to number sections
\setlength{\parindent}{0em} % horizontal indent for first line of paragraph
\setlength{\parskip}{1em} % vertical space between paragraphs

\newcommand{\textdesc}[1]{\textit{\textbf{#1}}}
\newcommand{\descitem}[1]{\item \textdesc{#1}}

\title{\documenttitle\\\scalebox{0.85}{\documentsubtitle}}
\author{Aslak Johansen \href{mailto:asjo@mmmi.sdu.dk}{asjo@mmmi.sdu.dk}\\Aisha Umair \href{mailto:aiu@mmmi.sdu.dk}{aiu@mmmi.sdu.dk}}

\begin{document}

\maketitle
\setcounter{tocdepth}{2}
\tableofcontentswrapper


\section*{Formål og Mål}

Formålet med iteration \#1 og \#2 af programudviklingen er at udføre selve udviklingsarbejdet. I skal nå så langt med både udviklingsarbejdet og udarbejdelsen af portfolien, at I får en forståelse for slutresultatet. Når I er færdige med iteration \#1 og \#2, så har projektgruppen lavet og fået godkendt:

\begin{itemize}
  \item en første udgave af produktet (kode)
  \item en første udgave af projektrapporten (begrænset udkast)
  \item en revideret samarbejdsaftale (efter behov)
  \item en revideret vejlederaftale (efter behov)
  \item en logbog, der fortsat dokumenterer gruppens arbejde
\end{itemize}

\section*{Opgaver i Fasen}

\subsection*{Problemanalyse}

I programudviklingen sker der en iterativ udvikling af jeres produkt baseret på jeres projektgrundlag, dvs. på basis af jeres problemformulering, med de underspørgsmål, den afgrænsning, de metoder og indenfor den tidsramme, som I har beskrevet.

Hensigten med første iteration er at få udarbejdet en kerneudgave af jeres produkt med en tekstbaseret brugergrænseflade (Command Line Interface) samt at få et første begrænset udkast til projektrapporten. I iteration \#2 udvæles der elementer at arbejde på som enten ikke er blevet ordentligt færdige, eller som I sprang over i iteration \#1.

\subsection*{Planlægning}

Projektgruppen laver en detaljeret planlægning af iteration \#1. Planlægningen bygger på gruppens projektgrundlag og i planlægningen respekteres rammeplan, frister og faste møder, online kurset i problemorienteret projektarbejde og det øvrige skema.

Projektarbejdet foregår fortsat uge for uge med fast projektarbejdsdag om tirsdagen.

\subsection*{Projektrapport}

Benyt modul 09 "Projektrapporten" i ProOnline til at forstå hvordan en projektrapport skrives på ingeniøruddannelserne og til at støtte jer i skrivningen af jeres portfolie.

\subsection*{Projektseminar og Aflevering}

Ved afslutningen af iterationen præsenteres projektet på et projektseminar og der afleveres udkast til portfolie og samt kode. Se nærmere om kravene til afleveringen i dokumentet \say{Krav til Projektaflevering}.

\section*{Aktiviteter i Fasen}

\begin{longtable}{|r|l|p{.6\textwidth}|l|}
  \hline
  \emph{Uge} & \emph{Aktivitet} & \emph{Beskrivelse} & \emph{Date} \\
  \hline
  41 & Projekt & Vejlederseminar + Projektvejledning i grupperne & Oct 10 \\
  \hline
  41 & ProOnline & \begin{itemize}[noitemsep,leftmargin=*,topsep=0pt,partopsep=0pt]

  \item 09 Projektrapporten og professionel formidling

\end{itemize} & Oct 13 \\
  \hline
  43 & Projekt & Projektgruppe vejledermøde & Oct 24 \\
  \hline
  43 & Universitet & Årsfest & Oct 27 \\
  \hline
  44 & Projekt & Projektgruppe vejledermøde & Oct 31 \\
  \hline
  45 & Projekt & \textbf{Midtvejsseminar i klasserne:} Præsentation af projekt

Aflevering af rapport samt kode efter iteration \#1 på itslearning & Nov 7 \\
  \hline
  46 & Projekt & Projektgruppe vejledermøde & Nov 14 \\
  \hline
  47 & Projekt & Projektgruppe vejledermøde & Nov 21 \\
  \hline
  47 & ProOnline & \begin{itemize}[noitemsep,leftmargin=*,topsep=0pt,partopsep=0pt]

  \item 10 Projekteksamen

\end{itemize} & Nov 24 \\
  \hline
  48 & Projekt & Projektgruppe vejledermøde & Nov 28 \\
  \hline
\end{longtable}


\section*{Materialer}

Materialer der er særligt vigtige i problemanalysefasen:
\begin{itemize}
  \item Projektgruppens projektgrundlag/poster
  \item Samarbejdsaftale, vejlederaftale, gruppelog
  \item Projektbeskrivelse\footnote{Under \say{Semesterprojekt og Projektbeskrivelse} på itslearning.}
  \item Semesterplan\footnote{Under \say{General Course/Semester Information} på itslearning.}
  \item Beskrivelsen af sidste fase\footnote{Under \say{02 Problemanalysefasen} på itslearning.}
%  \item Projektgrundlag (poster) - excel
\end{itemize}

\end{document}
