\documentclass[a4paper, oneside]{memoir}
\usepackage[danish]{babel} % load typographical rules for the english language
\usepackage{graphics} % for \scalebox
\usepackage{hyperref} % for \href
\usepackage{xcolor} % for text color
\usepackage{enumitem} % for ordered and unordered list
\usepackage{graphicx} % for images
\usepackage{pdfpages} % for including pdfs
\usepackage{footnote} % for footnotes
\usepackage{longtable} % for tabular environment that spans multiple pages and supports footnotes
\usepackage{colortbl} % for cell coloring
\usepackage{multirow} % for \multicolumn

% https://github.com/latex3/babel/issues/51
\makeatletter\AtBeginDocument{\let\@elt\relax}\makeatother

% styling
\setsecnumdepth{subsubsection} % how deep to number sections
\setlength{\parindent}{0em} % horizontal indent for first line of paragraph
\setlength{\parskip}{1em} % vertical space between paragraphs

\newcommand{\textdesc}[1]{\textit{\textbf{#1}}}
\newcommand{\descitem}[1]{\item \textdesc{#1}}

\title{\documenttitle\\\scalebox{0.85}{\documentsubtitle}}
\author{Aslak Johansen \href{mailto:asjo@mmmi.sdu.dk}{asjo@mmmi.sdu.dk}\\Aisha Umair \href{mailto:aiu@mmmi.sdu.dk}{aiu@mmmi.sdu.dk}}

\begin{document}

\maketitle
\setcounter{tocdepth}{2}
\tableofcontentswrapper

\textbf{Eksempel på rubric}

    \begin{center}
  \begin{tabular}{|p{2.5cm}| p{4cm}|p {4cm}|p {4cm}|}
    \hline
  \textbf{Rubric} & \textbf{Fremragende} & \textbf{Kompetent} & \textbf{Ikke færdigt (På begynderstadiet)}
 \\
    \hline
  \textbf{Samarbejde (30 point)} & Gruppen har arbejdet godt sammen for at nå målene. Hvert medlem har bidraget på værdifuld måde til projektet. Alt tyder på en høj grad af gensidig respekt og samarbejde & Gruppen har arbejdet godt sammen det meste af tiden, med kun et par tilfælde af sammenbrud i kommunikation eller svigt i samarbejdet. Det ser ud som gruppemedlemmerne i det store og hele har respekteret hinanden. & Gruppen har ikke samarbejdet og kommunikeret godt. Nogle af gruppens medlemmer har arbejdet for sig selv uden hensyntagen til mål eller prioriteringer. 
  Mangel på gensidig respekt blev noteret flere gange.
 \\
    \hline
 \textbf{Præsentation af resultater (30 point)} & Præsentationen gav dyb indsigt i det produkt, der er under udvikling. Præsentationen viste også, at der er gjort en stor indsats for at bryde nye veje og at opbygge engagement omkring produktet. & Præsentationen gav nogen indsigt i det produkt, der er under udvikling. Præsentationen viste også, at der er gjort noget for at opbygge engagement omkring produktet. & Præsentationen var ikke komplet og gav ingen betydende indsigt i det produkt, der er under udvikling. Præsentationen viste også, at der kun er gjort meget lidt for at opbygge engagement omkring produktet.
    \\
    \hline
\textbf{Præsentations indhold og kreativitet (40 point)} & Præsentationen var fantasifuld og formidlede effektivt gruppens ideer til publikum. & De anvendte præsentationsteknikker var effektive til at formidle hovedideer, men lidt fantasiløse. & Præsentationen kunne ikke fange publikums interesse og skabte forvirring om, hvad der blev formidlet.
     \\
    \hline
    \end{tabular}
    \end{center}
\end{document}