\documentclass[t, aspectratio=169]{beamer}

\usepackage[english]{babel}
\usepackage[utf8]{inputenc}
\usepackage{amsthm}
\usepackage{amsmath}
\usepackage{amsfonts}
\usepackage{graphicx}
\usepackage{tikz} % for diagrams
\usetikzlibrary[positioning]
\usetikzlibrary[fit]
\usetikzlibrary{patterns}
\usetikzlibrary{patterns.meta}
\usetikzlibrary{shapes.geometric}
\usetikzlibrary{shapes.arrows}
\usetikzlibrary{shapes}
\usetikzlibrary{arrows.meta}
\usetikzlibrary{calc}
\usetikzlibrary{tikzmark}
\usetikzlibrary{datavisualization}
\usetikzlibrary{datavisualization.formats.functions}
\usetikzlibrary{tikzmark}
%\usetikzlibrary{uml}
%\usepackage{tikz-uml} % https://perso.ensta-paris.fr/~kielbasi/tikzuml/index.php
\usepackage{pgf-umlcd}
\usepackage{pgfplots} % for plotting
\usepackage{adjustbox}
\usepackage{graphics}
\usepackage{tikz}
\usepackage{hyperref}
\usepackage{xspace} % for xspace
\usepackage{color} % for colored text
\usepackage{color, colortbl} % for colored tables
\usepackage{xcolor} % for colored cells in tables
\usepackage{fancyvrb} % for different fontsize in verbatim
\usepackage[T1]{fontenc} % for |
\usepackage{qrcode} % for QR codes
\usepackage{verbatim} % for verbatiminput
\usepackage{multicol} % for multicols
\usepackage{multirow} % for multirow
\usepackage{ragged2e} % for flushright environment
\usepackage{mathtools} % for mathmakebox
\usepackage{array} % for new array implementation: http://mirrors.ibiblio.org/CTAN/macros/latex/required/tools/array.pdf
\usepackage{dirtytalk} % for \say quotation
\usepackage{MnSymbol,wasysym} % for smileys

\setcounter{tocdepth}{4}
\setcounter{secnumdepth}{4}

\newcommand{\hlthing}[1]{\textcolor{orange}{#1}}
\newcommand{\hlcomment}[1]{\textcolor{blue}{#1}}
\newcommand{\hldata}[1]{\textcolor{teal}{#1}}

\newcommand{\quoted}[1]{\textsl{\say{#1}}}

% tikz configuration
\usetikzlibrary{arrows.meta, matrix, decorations.pathreplacing}
\usetikzlibrary{positioning}

\definecolor{aqua}{rgb}{0.878, 1.0, 1.0} % matches the default line highlight of minted

% sparql highlighting
\usepackage[cache=false]{minted}            % code inclusion
\newminted{sparql}{xleftmargin=1em,mathescape, numbersep=5pt, frame=lines, framesep=2mm, fontsize=\scriptsize, linenos}

\usepackage{pifont}%http://ctan.org/pkg/pifont
% http://tex.stackexchange.com/questions/42619/x-mark-to-match-checkmark
\newcommand{\cmark}{\text{\ding{51}}}
\newcommand{\xmark}{\text{\ding{55}}}
\newcommand{\change}{\text{\ding{46}}}

\newcommand{\valuename}[1]{\texttt{#1}}
\newcommand{\varname}[1]{\texttt{\textcolor{teal}{#1}}}
\newcommand{\typename}[1]{\texttt{\textcolor{purple}{#1}}}
\newcommand{\classname}[1]{\typename{#1}}
\newcommand{\interfacename}[1]{\typename{#1}}
\newcommand{\exceptionname}[1]{\classname{#1}}
\newcommand{\methodname}[1]{\texttt{\textcolor{blue}{#1}}}
\newcommand{\funcname}[1]{\methodname{#1}}
\newcommand{\packagename}[1]{\texttt{#1}}
\newcommand{\filename}[1]{\texttt{#1}}
\newcommand{\keywordname}[1]{\texttt{#1}}

\newcommand{\textdesc}[1]{\textit{\textbf{#1}}}
\newcommand{\descitem}[1]{\item \textdesc{#1}}
\newcommand{\includeSVG}[1]{
\includegraphics[scale=1.0]{./figs/#1.pdf}
}
\newcommand{\includeSVGfs}[1]{
\includegraphics[width=\paperwidth]{./figs/#1.pdf}
}
\newcommand{\includeBitmap}[2]{
\includegraphics[width=#2]{./figs/#1}
}

\newenvironment{inspiration}[1]
{
    \begin{center}
    \newcommand{\temp}{#1}
    \begin{minipage}{0.9\textwidth}
}
{
    \\
    \raggedright{-- \temp}
    \end{minipage}
    \end{center}
}

%% bypass highlighting of syntax errors in minted
%\renewcommand{\fcolorbox}[4][]{#4}

% remove paragraph indent
\setlength{\parindent}{0in}

% remove footer
\setbeamertemplate{footline}[page number]{} % gets rid of bottom navigation bars
\setbeamertemplate{navigation symbols}{} % gets rid of navigation symbols
\usecolortheme{default}
\definecolor{structurecolor}{RGB}{21,66,129}
\setbeamercolor{structure}{fg=structurecolor}
\setbeamertemplate{footline}{}

% PDF settings
\hypersetup{
pdftitle={Software Engineering og Software Teknologi - Introduktion til Semesteret},
pdfauthor={Aslak Johansen, Aisha Umair},
pdfsubject={},
pdfkeywords={Software Engineering, Softwareteknologi, Semester}
}

\title{Software Engineering og Softwareteknologi \\ \scalebox{0.9}{Introduktion til Semesteret}}
\author{
Aslak Johansen \hspace{1mm} \href{mailto:asjo@mmmi.sdu.dk}{asjo@mmmi.sdu.dk} \\
Aisha Umair \hspace{1mm} \href{mailto:aiu@mmmi.sdu.dk}{aiu@mmmi.sdu.dk}
}
\logo{\vspace{-0.5mm}\includegraphics[width=15mm]{figs/SDU.pdf}\hspace{0.5mm}}

\begin{document}

%%%%%%%%%%%%%%%%%%%%%%%%%%%%%%%%%%%%%%%%%%%%%%%%%%%%%%%%%%%%%%%
%%%%%%%%%%%%%%%%%%%%%%%%%%%%%%%%%%%%%%%%%%%%%%%%%%%%%%%%% Intro
\frame{\titlepage}
\logo{}

\section{Indhold}
\begin{frame}[fragile]
  \frametitle{Indhold}
  \vspace{2mm}
  \begin{columns}
    \begin{column}{0.55\textwidth}
      \begin{itemize}
        \item En syddansk ingeniør
        \item Første semester på Software Engineering og Softwareteknologi
        \item Semesterprojektet på 1. semester
        \item Problemorienteret projektarbejde
        \item Information om uddannelsesledelse og administration
        \item Introduktion til itslearning
        \item Online kursus i problemorienteret projektarbejde
        \item Afslutning
      \end{itemize}
    \end{column}
    \begin{column}{0.45\textwidth}
    \end{column}
  \end{columns}
  
  \begin{tikzpicture}[remember picture, overlay]
    \node[anchor=east] at ([xshift=1cm] current page.east) 
    {
        \includegraphics[height=\paperheight]{figs/image-001.png}
    };
  \end{tikzpicture}
\end{frame}

%%%%%%%%%%%%%%%%%%%%%%%%%%%%%%%%%%%%%%%%%%%%%%%%%%%%%%%%%%%%%%%
%%%%%%%%%%%%%%%%%%%%%%%%%%%%%%%%%%%%%%%% Den Syddanske Ingeniør
\section{Den Syddanske Ingeniør}
\begin{frame}
  \vspace{25mm}
  \begin{center}
    \Huge{Part 1:\\Den Syddanske Ingeniør}
  \end{center}
\end{frame}

\subsection{Hvad er en Ingeniør?}
\begin{frame}[fragile]
  \frametitle{Hvad er en Ingeniør?}
  \vspace{1mm}
  
\end{frame}

\subsection{Generelle Kompetencer}
\begin{frame}[fragile]
  \frametitle{Generelle Kompetencer}
  \vspace{1mm}
  
\end{frame}

\subsection{Hvordan opnås kompetencerne?}
\begin{frame}[fragile]
  \frametitle{Hvordan opnås kompetencerne?}
  \vspace{1mm}
  
\end{frame}

\subsection{ECTS Point}
\begin{frame}[fragile]
  \frametitle{}
  \vspace{1mm}
  
\end{frame}

\subsection{De to Første Semestre}
\begin{frame}[fragile]
  \frametitle{De to Første Semestre}
  \vspace{1mm}
  
\end{frame}

%%%%%%%%%%%%%%%%%%%%%%%%%%%%%%%%%%%%%%%%%%%%%%%%%%%%%%%%%%%%%%%
%%%%%%%%%%%%%%%%%%%%%%%%%%%%%%%%%%%%%%%%%%% Det Første Semester
\section{Det Første Semester}
\begin{frame}
  \vspace{25mm}
  \begin{center}
    \Huge{Part 2:\\Det Første Semester}
  \end{center}
\end{frame}

\subsection{Det Første Semester}
\begin{frame}[fragile]
  \frametitle{Det Første Semester}
  \vspace{1mm}
  
\end{frame}

\subsection{Undervisere}
\begin{frame}[fragile]
  \frametitle{Undervisere}
  \vspace{1mm}
  
\end{frame}

\subsection{Kompetencemål}
\begin{frame}[fragile]
  \frametitle{Kompetencemål}
  \vspace{1mm}
  
\end{frame}


\subsection{Semesterprojektet}
\begin{frame}[fragile]
  \frametitle{Semesterprojektet}
  \vspace{1mm}
  
\end{frame}


\subsection{Semesterprojektets Faser}
\begin{frame}[fragile]
  \frametitle{Semesterprojektets Faser}
  \vspace{1mm}
  
\end{frame}


\subsection{Projektarbejdet}
\begin{frame}[fragile]
  \frametitle{Projektarbejdet}
  \vspace{1mm}
  
\end{frame}

\subsection{Tænk - Snak - Del}
\begin{frame}[fragile]
  \frametitle{Tænk - Snak - Del}
  \vspace{1mm}
  
\end{frame}

%%%%%%%%%%%%%%%%%%%%%%%%%%%%%%%%%%%%%%%%%%%%%%%%%%%%%%%%%%%%%%%
%%%%%%%%%%%%%%%%%%%%%%%%%%%%%%%%%%%%%%%%%%%%%%%%%%% itslearning
\section{itslearning}
\begin{frame}
  \vspace{25mm}
  \begin{center}
    \Huge{Part 3:\\itslearning}
  \end{center}
\end{frame}

\subsection{itslearning}
\begin{frame}[fragile]
  \frametitle{itslearning}
  \vspace{1mm}
  
\end{frame}

\subsection{What is itslearning?}
\begin{frame}[fragile]
  \frametitle{What is itslearning?}
  \vspace{1mm}
  
\end{frame}

\subsection{Features of itslearning}
\begin{frame}[fragile]
  \frametitle{Features of itslearning}
  \vspace{1mm}
  
\end{frame}

\subsection{Home Page: Courses (tab)}
\begin{frame}[fragile]
  \frametitle{Home Page: Courses (tab)}
  \vspace{1mm}
  
\end{frame}

\subsection{Home Page: Updates (tab)}
\begin{frame}[fragile]
  \frametitle{Home Page: Updates (tab)}
  \vspace{1mm}
  
\end{frame}

\subsection{Home Page: Courses (tab)}
\begin{frame}[fragile]
  \frametitle{Home Page: Courses (tab)}
  \vspace{1mm}
  
\end{frame}

\subsection{Semesterprojekt: Front Page}
\begin{frame}[fragile]
  \frametitle{}
  \vspace{1mm}
  
\end{frame}

\subsection{Plans}
\begin{frame}[fragile]
  \frametitle{Plans}
  \vspace{1mm}
  
\end{frame}

\subsection{Resources}
\begin{frame}[fragile]
  \frametitle{Resources}
  \vspace{1mm}
  
\end{frame}

\subsection{Status and Follow-Up}
\begin{frame}[fragile]
  \frametitle{Status and Follow-Up}
  \vspace{1mm}
  
\end{frame}

\subsection{Semester Projekt: Course Structure}
\begin{frame}[fragile]
  \frametitle{Semester Projekt: Course Structure}
  \vspace{1mm}
  
\end{frame}

%%%%%%%%%%%%%%%%%%%%%%%%%%%%%%%%%%%%%%%%%%%%%%%%%%%%%%%%%%%%%%%
%%%%%%%%%%%%%%%%%%%%%%%%%%%%%%%%%%%%%%%%%%%%%%%%%%%%% ProOnline
\section{ProOnline}
\begin{frame}
  \vspace{25mm}
  \begin{center}
    \Huge{Part 3:\\ProOnline}
  \end{center}
\end{frame}

\subsection{Home Page: Courses (tab)}
\begin{frame}[fragile]
  \frametitle{Home Page: Courses (tab)}
  \vspace{1mm}
  
\end{frame}

\subsection{Problemorienteret Projektarbejde - et Online Kursus}
\begin{frame}[fragile]
  \frametitle{Problemorienteret Projektarbejde - et Online Kursus}
  \vspace{1mm}
  
\end{frame}

\subsection{Problemorienteret Projektarbejde - et Online Kursus}
\begin{frame}[fragile]
  \frametitle{Problemorienteret Projektarbejde - et Online Kursus}
  \vspace{1mm}
  
\end{frame}

\subsection{Problemorienteret Projektarbejde - et Online Kursus}
\begin{frame}[fragile]
  \frametitle{Problemorienteret Projektarbejde - et Online Kursus}
  \vspace{1mm}
  
\end{frame}

\subsection{Demo}
\begin{frame}[fragile]
  \frametitle{Demo}
  \vspace{1mm}
  
\end{frame}

\subsection{Næste Skridt}
\begin{frame}[fragile]
  \frametitle{}
  \vspace{1mm}
  
\end{frame}

\subsection{Skema}
\begin{frame}[fragile]
  \frametitle{Skema}
  \vspace{1mm}
  
\end{frame}

\subsection{How to Uni}
\begin{frame}[fragile]
  \frametitle{How to Uni}
  \vspace{1mm}
  
\end{frame}

%%%%%%%%%%%%%%%%%%%%%%%%%%%%%%%%%%%%%%%%%%%%%%%%%%%%%%%%%%%%%%%
%%%%%%%%%%%%%%%%%%%%%%%%%%%%%%%%%%%%%%%%%%%%%%%%%%%%% Thank You
\section{Thank You}
\begin{frame}
  \vspace{25mm}
  \begin{center}
    \Huge{Thank You\\Wish you Good Luck \smiley{}}
  \end{center}
\end{frame}

\end{document}
