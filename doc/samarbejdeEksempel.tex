\documentclass[a4paper, oneside]{memoir}
\usepackage[english]{babel} % load typographical rules for the english language
\usepackage{graphics} % for \scalebox
\usepackage{hyperref} % for \href
\usepackage{xcolor} % for text color
\usepackage{enumitem} % for ordered and unordered list
\usepackage{graphicx} % for images
\usepackage{pdfpages} % for including pdfs

% https://github.com/latex3/babel/issues/51
\makeatletter\AtBeginDocument{\let\@elt\relax}\makeatother

% styling
\setsecnumdepth{subsubsection} % how deep to number sections
\setlength{\parindent}{0em} % horizontal indent for first line of paragraph
\setlength{\parskip}{1em} % vertical space between paragraphs

\newcommand{\textdesc}[1]{\textit{\textbf{#1}}}
\newcommand{\descitem}[1]{\item \textdesc{#1}}

\title{Project Description}
\author{Aslak Johansen}

\begin{document}

\maketitle
\setcounter{tocdepth}{2}
\tableofcontents


\chapter{Samarbejde: Eksempel på en samarbejdsaftale}

%%%%%%%%%%%%%%%%%%%%%%%%%%%%%%%%%%%%%%%%%%%%%%%%%%%%%%%%%%%%%%%%%%%%%%%%%%%%%%%%
%%%%%%%%%%%%%%%%%%%%%%%%%%%%%%%%%%%%%%%%%%%%%%%%%%%%%%%%%%%%%%%%%%%%%%%%%%% Forventninger til og mål med projektet
\section{Forventninger til og mål med projektet}
\begin{itemize}
\item Det er et fælles mål blandt gruppemedlemmerne at blive bedre til fagets indhold. Alle ønsker, at alle får en forståelse for det hele, så vi fælles niveau i afslutningen af projektet.
\item Gruppemedlemmerne bestræber så vidt muligt at fremstille et projekt som alle er tilfredse med.
\item Gruppens medlemmer skal så vidt muligt gøre deres bedste. Dog er der en grænse for hvad hver især kan præstere, og man kan selvfølgelig ikke kræve mere. Derfor er det ikke nødvendigt at score topkarakterer, bare gruppemedlemmerne gør sit bedste. Selvom topkarakterer selvfølgelig er hvad vi alle håber på.
\item Det forventes, at alle medlemmer af gruppen leverer det bedst mulige materiale, når personen har fået en opgave. Herunder forventes det, at personen selv har læst korrektur på sit materiale.
\end{itemize}

%%%%%%%%%%%%%%%%%%%%%%%%%%%%%%%%%%%%%%%%%%%%%%%%%%%%%%%%%%%%%%%%%%%%%%%%%%%%%%%%
%%%%%%%%%%%%%%%%%%%%%%%%%%%%%%%%%%%%%%%%%%%%%%%%%%%%%%%%%%%%%%%%%%%%%%%%%%% Hvad motiverer den enkelte i gruppen til at gøre en indsats?

\section{Hvad motiverer den enkelte i gruppen til at gøre en indsats?}
\begin{itemize}
\item Ansvarsfølelsen over for de andre i gruppen gør at man yder en bedre præstation.
\item Den afsluttende karakter
\end{itemize}

%%%%%%%%%%%%%%%%%%%%%%%%%%%%%%%%%%%%%%%%%%%%%%%%%%%%%%%%%%%%%%%%%%%%%%%%%%%%%%%%
%%%%%%%%%%%%%%%%%%%%%%%%%%%%%%%%%%%%%%%%%%%%%%%%%%%%%%%%%%%%%%%%%%%%%%%%%%% Arbejdstider

\section{Arbejdstider}
\begin{itemize}
\item Møder klokken 8:30 hver tirsdag på grund af dumme bustider. Arbejder til klokken 16 medmindre gruppens medlemmer er enige om at tage tidligt fri.
\item Møder klokken 8:15 hver fredag medmindre der er dumme bustider. Vi arbejder selvfølgelig til forelæsningen begynder klokken 12.
\item Hvis der gives hjemmearbejde, forventes det, at man afsætter tid derhjemme til at lave det.
\end{itemize}


%%%%%%%%%%%%%%%%%%%%%%%%%%%%%%%%%%%%%%%%%%%%%%%%%%%%%%%%%%%%%%%%%%%%%%%%%%%%%%%%
%%%%%%%%%%%%%%%%%%%%%%%%%%%%%%%%%%%%%%%%%%%%%%%%%%%%%%%%%%%%%%%%%%%%%%%%%%% Gruppemøder

\section{Gruppemøder}
\begin{itemize}
\item Tirsdag og fredag mødes vi. Eventuelt kan vi arrangere møder uden for skoletiden og arbejde hvis det bliver nødvendigt.
\item Hvis man kommer for sent: Hvis man ikke overholder de planlagte mødetider, skal der pålægges en form for straf medmindre årsagen kan begrundes godt eller hvis der bare er tale om et par minutter. Dette kan være i form af kage eller bajer. Hvis man ved, at man kommer for sent, skal man skrive en besked til gruppen.
\item Hvis et gruppemedlem ikke møder op til et planlagt møde, skal man skrive en besked til gruppen før mødet starter. Her kan der pålægges en eventuel straf, hvis årsagen ikke er begrundet tilstrækkeligt. Hvis et gruppemedlem gentager et sådant fravær flere gange, tages der en snak med gruppemedlemmet for at finde en løsning. Hvis det ikke er muligt at komme til enighed, må vejlederen involveres. Det samme gælder hvis man ikke laver det aftalte arbejde.
\item Det bestræbes at minimere overflødige diskussioner, men de tillades stadig i små mængder for at gøre arbejdsdagene udholdelige.
\end{itemize}


%%%%%%%%%%%%%%%%%%%%%%%%%%%%%%%%%%%%%%%%%%%%%%%%%%%%%%%%%%%%%%%%%%%%%%%%%%%%%%%%
%%%%%%%%%%%%%%%%%%%%%%%%%%%%%%%%%%%%%%%%%%%%%%%%%%%%%%%%%%%%%%%%%%%%%%%%%%% Organisering af møder - ordstyrer - referent – logbog

\section{Organisering af møder - ordstyrer - referent – logbog}
\begin{itemize}
\item Til hvert vejledermøde får ét gruppemedlem rollen som ordstyrer og en anden får rollen som referent. Rollerne som ordstyrer og referent følger det forudfremstillede skema.
\item X’et under ”Referent” og ”Ordstyrer” indikerer hvem der får tildelt disse roller. Efter hver gang rykkes de to x’er én plads ned, så rollerne går videre:
\begin{center}
  \begin{tabular}{|p{4cm}|p{2cm}|p{2cm}|}
    \hline
   &
    \textbf{Referent} & \textbf{Ordstyrer}
 \\
    \hline
  Anne Annesdatter & & X 
 \\
    \hline
     Jan Jansen & X & 
 \\
    \hline
     Mathias Mathiassen & &
 \\
    \hline
     Jens Jensen & &
 \\
    \hline
     Steen Steensen & & 
 \\
    \hline
    \end{tabular}
    \end{center}
\item Hver eneste møde beskrives i gruppens logbog. Både vejledermøder og arbejdsmøder. Hver arbejdsdag betragtes som mindst et arbejdsmøde. Rollen som logbogs-skriver følger det forudfremstillede skema:
\begin{center}
  \begin{tabular}{|p{4cm}|p{2cm}|p{2cm}|}
    \hline
   &
    \textbf{Logbog} 
 \\
    \hline
  Anne Annesdatter & X 
 \\
    \hline
     Jan Jansen &
 \\
    \hline
     Mathias Mathiassen &
 \\
    \hline
     Jens Jensen &
 \\
    \hline
     Steen Steensen &
 \\
    \hline
    \end{tabular}
    \end{center}
\item Logbogen indeholder
\begin{itemize}
\item alle indkaldelser med dagsorden
 \item alle referater med angivelse af hvem der var til stede og begrundelser for manglende tilstedeværelse
\end{itemize}
\end{itemize}


%%%%%%%%%%%%%%%%%%%%%%%%%%%%%%%%%%%%%%%%%%%%%%%%%%%%%%%%%%%%%%%%%%%%%%%%%%%%%%%%
%%%%%%%%%%%%%%%%%%%%%%%%%%%%%%%%%%%%%%%%%%%%%%%%%%%%%%%%%%%%%%%%%%%%%%%%%%% Fælles versus individuelt arbejde - hvor meget? – hvornår?

\section{Fælles versus individuelt arbejde - hvor meget? – hvornår?}
\begin{itemize}
\item Hver gang vi mødes på skolen, sætter vi alle os i samme lokale og arbejder fælles. Dette kan dog foregå på forskellige måder. Nogle gange kan vi arbejde sammen om en ting, og andre gange kan arbejdet deles op. Dog skal vi altid kunne have mulighed for kunne spørge hinanden til råds.
\item Individuelt arbejde aftales fra gang til gang og laves hjemme.
\end{itemize}

%%%%%%%%%%%%%%%%%%%%%%%%%%%%%%%%%%%%%%%%%%%%%%%%%%%%%%%%%%%%%%%%%%%%%%%%%%%%%%%%
%%%%%%%%%%%%%%%%%%%%%%%%%%%%%%%%%%%%%%%%%%%%%%%%%%%%%%%%%%%%%%%%%%%%%%%%%%% Arbejdsdeling i forhold til skriftligt arbejde?

\section{Arbejdsdeling i forhold til skriftligt arbejde?}
\begin{itemize}
\item De forskellige afsnit i rapporten fordeles blandt gruppens medlemmer for at gøre det skriftlige arbejde effektivt. Hvilke afsnit der skal fordeles til hvem tages løbende igennem processen.
\end{itemize}

%%%%%%%%%%%%%%%%%%%%%%%%%%%%%%%%%%%%%%%%%%%%%%%%%%%%%%%%%%%%%%%%%%%%%%%%%%%%%%%%
%%%%%%%%%%%%%%%%%%%%%%%%%%%%%%%%%%%%%%%%%%%%%%%%%%%%%%%%%%%%%%%%%%%%%%%%%%% Vejledermøder - hvor tit? - Forberedelse?

\section{Vejledermøder - hvor tit? - Forberedelse?}
\begin{itemize}
\item Der afholdes ét vejledermøde hver uge for at sikre at vi altid er på rette spor.
\item Flere / færre vejledermøder planlægges efter behov.
\item Gruppen har pligt til at sende en mødeindkaldelse med lokale og link til dagsorden og materialer via outlook til vejlederen senest kl. 12 dagen før materialet skal behandles.
\end{itemize}


%%%%%%%%%%%%%%%%%%%%%%%%%%%%%%%%%%%%%%%%%%%%%%%%%%%%%%%%%%%%%%%%%%%%%%%%%%%%%%%%
%%%%%%%%%%%%%%%%%%%%%%%%%%%%%%%%%%%%%%%%%%%%%%%%%%%%%%%%%%%%%%%%%%%%%%%%%%% Kursusdeltagelse og opgaveløsning - individuelt eller fælles?


\section{Kursusdeltagelse og opgaveløsning - individuelt eller fælles?}
\begin{itemize}
\item Med mindre andet aftales, deltager alle gruppens medlemmer til alle forelæsninger.
\item Så vidt muligt skrives kommentarer i koden for at gøre den så forståelig som mulig.
\item Det forventes, at hvert gruppemedlem har læst op og fået styr på de emner vi arbejder med.
\end{itemize}


%%%%%%%%%%%%%%%%%%%%%%%%%%%%%%%%%%%%%%%%%%%%%%%%%%%%%%%%%%%%%%%%%%%%%%%%%%%%%%%%
%%%%%%%%%%%%%%%%%%%%%%%%%%%%%%%%%%%%%%%%%%%%%%%%%%%%%%%%%%%%%%%%%%%%%%%%%%% Deadlines

\section{Deadlines}
\begin{itemize}
\item Uge 41: Postersession med præsentation af projektgrundlag
\item Uge 46: Midtvejsseminar (Udkast til rapport og foreløbig kode
\item Uge 50 (14. december): Aflevering af kode
\item Uge 51 (21. december): Aflevering af rapport
\item Uge 1 - 4: EKSAMEN!!! ÅH NEJ DOG!.
\end{itemize}
Alle deadlines også indbyrdes aftalte deadlines overholdes. Er dette ikke muligt, skal det angives i logbogen hvorfor tidsgrænsen ikke er overholdt.


%%%%%%%%%%%%%%%%%%%%%%%%%%%%%%%%%%%%%%%%%%%%%%%%%%%%%%%%%%%%%%%%%%%%%%%%%%%%%%%%
%%%%%%%%%%%%%%%%%%%%%%%%%%%%%%%%%%%%%%%%%%%%%%%%%%%%%%%%%%%%%%%%%%%%%%%%%%% Brug og revision af samarbejdsaftalen

\section{Brug og revision af samarbejdsaftalen}
\begin{itemize}
\item Samarbejdsaftalen bruges løbende.
\item Aftalen tages i brug ved konflikter.
\item Aftalen tages op til behandling ved hver faseafslutning: Hvad aftalte vi? Hvad gjorde vi? Hvad skal vi blive ved med at gøre? Hvad skal vi holde op med at gøre? Hvad skal vi begynde at gøre?
\end{itemize}


%%%%%%%%%%%%%%%%%%%%%%%%%%%%%%%%%%%%%%%%%%%%%%%%%%%%%%%%%%%%%%%%%%%%%%%%%%%%%%%%
%%%%%%%%%%%%%%%%%%%%%%%%%%%%%%%%%%%%%%%%%%%%%%%%%%%%%%%%%%%%%%%%%%%%%%%%%%% Kontaktoplysninger

\section{Kontaktoplysninger}
Gruppe
\begin{itemize}
\item Anne Annesdatter
\begin{itemize}
\item Mail: anan19@student.sdu.dk
\item Telefon: 12345678
\end{itemize}
\item Jan Jansen
\begin{itemize}
\item Mail: jaja19@student.sdu.dk
\item Telefon: 23456781
\end{itemize}
\item Mathias Mathiassen
\begin{itemize}
\item Mail: jaja19@student.sdu.dk
\item Telefon: 34567812
\end{itemize}
\item Jens Jensen
\begin{itemize}
\item Mail: jaja19@student.sdu.dk
\item Telefon: 45678123
\end{itemize}
\item Steen Steensen
\begin{itemize}
\item Mail: jaja19@student.sdu.dk
\item Telefon: 56781234
\end{itemize}
\end{itemize}
Vejleder
\begin{itemize}
\item Amalie Amaliedatter
\begin{itemize}
\item Mail: amam@sdu.dk
\item Telefon: 6781234
\end{itemize}
\end{itemize}
Eksemplet er udarbejdet af en gruppe efteråret 2018. Navne er anonymiseret.
\end{document}