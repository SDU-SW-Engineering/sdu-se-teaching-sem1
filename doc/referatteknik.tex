\documentclass[a4paper, oneside]{memoir}
\usepackage[danish]{babel} % load typographical rules for the english language
\usepackage{graphics} % for \scalebox
\usepackage{hyperref} % for \href
\usepackage{xcolor} % for text color
\usepackage{enumitem} % for ordered and unordered list
\usepackage{graphicx} % for images
\usepackage{pdfpages} % for including pdfs
\usepackage{footnote} % for footnotes
\usepackage{longtable} % for tabular environment that spans multiple pages and supports footnotes
\usepackage{colortbl} % for cell coloring
\usepackage{multirow} % for \multicolumn

% https://github.com/latex3/babel/issues/51
\makeatletter\AtBeginDocument{\let\@elt\relax}\makeatother

% styling
\setsecnumdepth{subsubsection} % how deep to number sections
\setlength{\parindent}{0em} % horizontal indent for first line of paragraph
\setlength{\parskip}{1em} % vertical space between paragraphs

\newcommand{\textdesc}[1]{\textit{\textbf{#1}}}
\newcommand{\descitem}[1]{\item \textdesc{#1}}

\title{\documenttitle\\\scalebox{0.85}{\documentsubtitle}}
\author{Aslak Johansen \href{mailto:asjo@mmmi.sdu.dk}{asjo@mmmi.sdu.dk}\\Aisha Umair \href{mailto:aiu@mmmi.sdu.dk}{aiu@mmmi.sdu.dk}}

\begin{document}

\maketitle
\setcounter{tocdepth}{2}
\tableofcontentswrapper

\begin{savenotes}
\chapter{Notat- og referatteknik}
[..] Man vil undertiden komme ud for at skulle tage notater eller lave mødereferater.

\\ Notatteknik\footnote{Denne vejledning er hentet fra http://forlaget94.dk/download/62}. bruges, når tingene opstår spontant. I undervisningslokalet, i gruppearbejdet, ved selvstudiet, på arbejdspladsen eller på kongressen. Men også som hjælp til at huske, hvad der skal refereres i et mødereferat.

\\ Referatteknik bruges, når man skal beskrive og referere et projektforløb, en aktion, et møde eller et handlingsforløb. Referatet kan være baseret på et notat fra gruppemedlemmer eller fra en selv.
\end{savenotes}
%%%%%%%%%%%%%%%%%%%%%%%%%%%%%%%%%%%%%%%%%%%%%%%%%%%%%%%%%%%%%%%%%%%%%%%%%%%%%%%%
%%%%%%%%%%%%%%%%%%%%%%%%%%%%%%%%%%%%%%%%%%%%%%%%%%%%%%%%%%%%%%%%%%%%%%%%%%% Notatteknik


\section{Notatteknik}
Det kan være godt at kende til notatteknik, inden man siger ja til at tage referat fra et møde eller lignende. Notatteknik handler ikke om at skrive alt ned, men om at skrive det vigtigste ned. Lige præcis det, der kan støtte hukommelsen, når referatet senere skal skrives.

\\ Notatpapiret kan være en stor hjælp, når der skal tages referat. Hvis papiret ikke er fortrykt med „rigtig“ inddeling, kan man fremstille et notatark ved hjælp af nogle streger – del fx papiret op i felter som vist i figuren.

\\ Figur: Notatpapirets opdeling

%%%%%%%%%%%%%%%%%%%%%%%%%%%%%%%%%%%%%%%%%%%%%%%%%%%%%%%%%%%%%%%%%%%%%%%%%%%%%%%%
%%%%%%%%%%%%%%%%%%%%%%%%%%%%%%%%%%%%%%%%%%%%%%%%%%%%%%%%%%%%%%%%%%%%%%%%%%% Lytte – tænke – noter


\section{Lytte – tænke – noter}
Inddelingen er med til at gøre notatet overskueligt. Tekst og nøgleord skal noteres gradvist: Man skal ikke straks begynde at skrive, men i stedet lytte, forstå, forkorte, formulere og notere. Det gælder altså om at
\begin{itemize}
    \item lytte: Hvad er det, der siges?
    \item tænke: 
       \begin{itemize}[label={}]
          \item Vurdere – hvad menes der?
          \item  Sortere – hvad skal med?
          \item Formulere – hvordan udtrykkes det?
       \end{itemize}
         
          
     \item notere: Notere – skrive ned – og lytte videre; være klar til at tænke igen – vurdere, sortere og formulere.
\end{itemize}
Man noterer altså i „ryk“; men først når man har lyttet og forstået.

%%%%%%%%%%%%%%%%%%%%%%%%%%%%%%%%%%%%%%%%%%%%%%%%%%%%%%%%%%%%%%%%%%%%%%%%%%%%%%%%
%%%%%%%%%%%%%%%%%%%%%%%%%%%%%%%%%%%%%%%%%%%%%%%%%%%%%%%%%%%%%%%%%%%%%%%%%%% Notatsprog


\section{Notatsprog}
Nedskrivningen skal kunne foregå i en rasende fart, være kort og let at forstå. Det er ikke nemt at forene, men rutine og øvelse hjælper – ligesom det kan være hensigtsmæssigt at huske nedenstående tommelfingerregler:
\\
\\ Udelad detaljer: Medtag det generelle:
\\ Eksempler \quad Hovedlinjerne
\\ Gentagelser \quad Det vigtige
\\ Fyldord
\\ Det specielle
\\
\\ For at bevare hurtigheden er det nyttigt at udvikle et særligt notatsprog. Fx ved at gøre brug af nøgleord, forkortelser, tegn og afkortede sætninger.

\\ Nøgleordene er i denne forbindelse vigtige. Nøgleordene er de ord, der sætter en tankerække i gang. En tankerække af de associationer, ordet vækker. Derfor vil nøgleordene altid være dem, der rummer:
\begin{itemize}
    \item Begreber.
    \item Særpræg.
    \item Hovedlinjer.
    \item Det vigtige.
\end{itemize}  
Forkortelser og tegn – husk at notere hele ordet første gang. Samtidig bør forkortelsen anføres i en parentes. Man kan med fordel lave sine egne tegn og forkortelser. Her er et par eksempler:
\begin{center}
  \begin{tabular}{p{2cm} p{6cm}}
  + & Positiv, bekræftende, yderligere
  \\  – & Negativ, benægtende, ikke
  \\ ? & Spørgsmål, tvivlsomt, uafklaret
  \\ = & Lig med, på samme måde, analogt
  \\ mao & Med andre ord
  \\ pdes &  På den ene side
  \\ pdas &  På den anden side
  \\ fsv & For så vidt
  \\ fx & For eksempel
  \\ iht & I henhold til
 
    \end{tabular}
    \end{center}
    Noterne fra et møde skal være grundige og meget mere fyldige end det referat, de skal danne grundlag for. Det er bedre at notere for meget end at stå og mangle noget senere. Skriv referatet kort efter mødet, så notater og hukommelse kan supplere hinanden.

%%%%%%%%%%%%%%%%%%%%%%%%%%%%%%%%%%%%%%%%%%%%%%%%%%%%%%%%%%%%%%%%%%%%%%%%%%%%%%%%
%%%%%%%%%%%%%%%%%%%%%%%%%%%%%%%%%%%%%%%%%%%%%%%%%%%%%%%%%%%%%%%%%%%%%%%%%%% Referatteknik

\section{Referatteknik}
Referatets form skal passe til formålet, så de forskellige informationer kan bruges til fx til at gengive og informere om det vigtigste indhold og de væsentligste resultater af et møde. Samtidig bliver referatet det skriftlige bevis for, hvad der er foregået på mødet. Derfor skal et mødereferat også kunne bruges til at fastholde mødets forløb, erindre deltagerne, orientere ikke-deltagere og danne grundlag for handling. Samtidig virker referatet som kommunikationsmiddel, informationsgrundlag, arbejdsgrundlag, juridisk dokument og historisk dokument.

%%%%%%%%%%%%%%%%%%%%%%%%%%%%%%%%%%%%%%%%%%%%%%%%%%%%%%%%%%%%%%%%%%%%%%%%%%%%%%%%
%%%%%%%%%%%%%%%%%%%%%%%%%%%%%%%%%%%%%%%%%%%%%%%%%%%%%%%%%%%%%%%%%%%%%%%%%%% Referentens arbejdsopgaver


\section{Referentens arbejdsopgaver}
Jobbet som referent har tre faser:
\begin{itemize}
    \item Før mødet.
    \item Selve mødet.
    \item Efter mødet.
\end{itemize}
Før mødet: Forberedelse af notatpapiret – herunder notatpapirets felt 2. Feltet kan indeholde grundoplysninger efter behov; eksempelvis dato, deltagerinitialer (mødeleder, referent og øvrige deltagere), sted, sidenummer, emne, sagsnummer, fraværende med mere.

\\ Som referent bør man sætte sig godt ind i mødets emne. Det betyder blandt andet, at man bør kende de vigtigste fagudtryk og eventuelle fremmedord, som deltagerkredsen normalt bruger. En referent skal være i stand til at gengive meningen i det sagte og ikke blot ordene. Man kan eventuelt bede mødelederen om en liste over de mest brugte fagudtryk.

\\ Under mødet: Forekommer der fremmedord, fagudtryk eller uforståelige udsagn, så notér dem ned så nøjagtigt som muligt. Spørg straks eller lige efter mødet til disse udtryk.

\\ Referatskrivning: Med afsæt i de noter, der er taget, skal man skrive referatet. Det er en fordel at foretage notaterne på samme blanket, som referatet skal skrives på. Derved kan den struktur, der anvendes i notaterne, lettere overføres til renskriften.


%%%%%%%%%%%%%%%%%%%%%%%%%%%%%%%%%%%%%%%%%%%%%%%%%%%%%%%%%%%%%%%%%%%%%%%%%%%%%%%%
%%%%%%%%%%%%%%%%%%%%%%%%%%%%%%%%%%%%%%%%%%%%%%%%%%%%%%%%%%%%%%%%%%%%%%%%%%% Indhold


\section{Indhold}
Hvad skal så med i et referat? Det afhænger naturligvis af, hvad der sker på mødet, og hvad referatet skal bruges til. Men som udgangspunkt bør et ordentligt mødereferat indeholde følgende punkter:
\begin{itemize}
    \item Mødetype: Afdelingsmøde, projektgruppemøde eller andet.
    \item Mødenummer: Specielt hvis mødet er et i en række – det letter den efterfølgende læsning af referatet.
    \item Sags-/journalnummer: Et bestemt projekt eller en opgave – og selvfølgelig et journalnummer, hvis der er et.
    \item Sted, dato, klokkeslæt.
    \item Mødedeltagere: Hvem var til stede, hvem var fraværende (gælder specielt til møder, hvor der er mødepligt), og hvem var referent?
    \item Dagsorden: Gør det lettere at henvise til, hvad der blev talt om.
    \item Materiale: Blev der uddelt materiale på mødet? Eller henviser en mødedeltager til tidligere udsendt materiale?
    \item Sagsbehandling: Hvad er der blevet sagt eller besluttet om det enkelte punkt?
    \item Tidspunkt for afslutning.
    \item Næste møde.
    \item Referatfordeling: Hvem skal have en kopi af referatet?
\end{itemize}

%%%%%%%%%%%%%%%%%%%%%%%%%%%%%%%%%%%%%%%%%%%%%%%%%%%%%%%%%%%%%%%%%%%%%%%%%%%%%%%%
%%%%%%%%%%%%%%%%%%%%%%%%%%%%%%%%%%%%%%%%%%%%%%%%%%%%%%%%%%%%%%%%%%%%%%%%%%% Sproget i referatet


\section{Sproget i referatet}
Sproget skal være letforståeligt for modtagergruppen. Det vil sige en neutral og sagligt præget fremstilling. At fremstillingen skal være neutral betyder dog ikke, at der ikke må anvendes udtryk og begreber – den terminologi – der er normal i sammenhængen. Ofte vil referatet nemlig ikke komme videre end til dem, der har samme faglige og professionelle baggrund. Det er derfor vigtigst, at det umiddelbart forstås og kan anvendes af de mennesker, der har brug for det. Der bør altid refereres loyalt og neutralt.

\\ Selve renskriften af referatet skal være systematisk og overskuelig. Læseren af referatet skal have det vigtigste med fra mødet – ikke gentagelser, anekdoter, det overflødige. Konklusionen skal bygges på logiske og klare præmisser. De fleste referater skrives i datid.

%%%%%%%%%%%%%%%%%%%%%%%%%%%%%%%%%%%%%%%%%%%%%%%%%%%%%%%%%%%%%%%%%%%%%%%%%%%%%%%%
%%%%%%%%%%%%%%%%%%%%%%%%%%%%%%%%%%%%%%%%%%%%%%%%%%%%%%%%%%%%%%%%%%%%%%%%%%% Sidste del af referentens arbejde


\section{Sidste del af referentens arbejde}
Eventuelle bilag til referatet sendes sammen med referatet. Referatet skal eventuelt registreres og arkiveres enten elektronisk eller manuelt.
\end{document}