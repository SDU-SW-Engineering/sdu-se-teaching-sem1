\documentclass[a4paper, oneside]{memoir}
\usepackage[danish]{babel} % load typographical rules for the english language
\usepackage{graphics} % for \scalebox
\usepackage{hyperref} % for \href
\usepackage{xcolor} % for text color
\usepackage{enumitem} % for ordered and unordered list
\usepackage{graphicx} % for images
\usepackage{pdfpages} % for including pdfs
\usepackage{footnote} % for footnotes
\usepackage{longtable} % for tabular environment that spans multiple pages and supports footnotes
\usepackage{colortbl} % for cell coloring
\usepackage{multirow} % for \multicolumn

% https://github.com/latex3/babel/issues/51
\makeatletter\AtBeginDocument{\let\@elt\relax}\makeatother

% styling
\setsecnumdepth{subsubsection} % how deep to number sections
\setlength{\parindent}{0em} % horizontal indent for first line of paragraph
\setlength{\parskip}{1em} % vertical space between paragraphs

\newcommand{\textdesc}[1]{\textit{\textbf{#1}}}
\newcommand{\descitem}[1]{\item \textdesc{#1}}

\title{\documenttitle\\\scalebox{0.85}{\documentsubtitle}}
\author{Aslak Johansen \href{mailto:asjo@mmmi.sdu.dk}{asjo@mmmi.sdu.dk}\\Aisha Umair \href{mailto:aiu@mmmi.sdu.dk}{aiu@mmmi.sdu.dk}}

\begin{document}

\maketitle
\setcounter{tocdepth}{2}
\tableofcontentswrapper


\section*{Formål og Mål}

Formålet med projektstarten er at få igangsat projektet samt etableret klasserne, projektgruppen og dens samarbejde. Når vi er færdige med projektstarten, så er der sket følgende:

\begin{itemize}
  \item I klasserne:
    \begin{itemize}
      \item Studerende har mødt hinanden og lært hinanden at kende.
      \item Studerende har fået en forståelse for rammerne for semesterprojektet.
    \end{itemize}
  \item I projektgrupperne:
    \begin{itemize}
      \item Grupperne har fået de første ideer til projektets problemstilling.
      \item Grupperne har fået de første ideer til softwareløsningen.
      \item Gruppen har lavet og fået godkendt:
        \begin{itemize}
          \item En samarbejdsaftale.
          \item En vejlederaftale.
          \item En logbog, til fastholdelse af gruppens arbejde.
        \end{itemize}
    \end{itemize}
\end{itemize}

\section*{Indhold}

\textsl{Projektstart er Takeoff og Etablering.}

\textcolor{purple}{Takeoff} består af information om projektet og gruppedannelse. Semesterkoordinatorerne og projektvejlederne står for takeoff, og informerer om semestret,  semesterprojektet og online kurset i problemorienteret projektarbejde på et klassemøde. De danner desuden projektgrupper og inviterer til de første projektseminarer (vejledermøder i klasserne).

Efter takeoff står projektgruppen selv for at gennemføre arbejdet i \textcolor{purple}{etablering} med vejledning fra vejlederen på projektgruppebasis. Arbejdet resulterer i en første forståelse af rammerne for projektet og udarbejdelse af en samarbejdsaftale og en vejlederaftale.

\section*{Oversigt over Aktiviteter}

\begin{longtable}{|r|l|p{.6\textwidth}|l|}
  \hline
  \emph{Uge} & \emph{Aktivitet} & \emph{Beskrivelse} & \emph{Date} \\
  \hline
  36 & ProOnline & \begin{itemize}[noitemsep,leftmargin=*,topsep=0pt,partopsep=0pt]

  \item 01 Introduktion til online kursus i problemorienteret projektarbejde

  \item 02 Projekter på ingeniøruddannelserne

\end{itemize} & Sep 5 \\
  \hline
  36 & Projekt & \textbf{Vejledningsseminar i klasserne:} Projektstart

\par

Dagsorden:

\begin{enumerate}[noitemsep,leftmargin=*,topsep=0pt,partopsep=0pt]

 \descitem{Velkomst ved vejleder}

 \descitem{Semesterprojektet og projektforløbet} Projektvejleder præsenterer projektet\footnote{Semesterhåndbog og projektbeskrivelse på itslearning.},  projektplanen\footnote{Semesterplan på itslearning.} og giver jer et eksempel fra et tidligere semester.  I får mulighed for at snakke med vejleder om, hvad meningen er med projektet, hvad det handler om, hvilken rolle verdensmålene spiller, og hvordan kommer projektet kommer til at foregå.

 \descitem{Diskussion om projektet og ideer til projektet} Projektet diskuteres i mindre grupper og efterfølgende plenumdiskussion.

 \descitem{Problemorienteret projektarbejde: Modul 01 og 02} Medbring jeres svar fra opgaverne i Modul 01 og 02 til viderebearbejdning i klassen.

 \descitem{Projekteksamen - læringsmål\footnote{\url{https://odin.sdu.dk/sitecore/index.php?a=searchfagbesk&bbcourseid=T510047101-1-E21&lang=da}} og rapportcheckliste\footnote{Se \textsl{"Krav til projektaflevering"} på itslearning under \textsl{"Projektafleveringfasen og Evaluationfasen"}}} Hvordan kommer eksamen til at foregå og hvad vil I blive bedømt på?

\end{enumerate} & Sep 6 \\
  \hline
  36 & ProOnline & \begin{itemize}[noitemsep,leftmargin=*,topsep=0pt,partopsep=0pt]

  \item 03 Samarbejde

\end{itemize} & Sep 9 \\
  \hline
  37 & Projekt & \textbf{Vejledningsseminar i klasserne:} Samarbejde

\par

Dagsorden:

\begin{enumerate}[noitemsep,leftmargin=*,topsep=0pt,partopsep=0pt]

  \descitem{Velkomst ved vejleder} 

  \descitem{Problemorienteret projektarbejde: Modul 03 Samarbejde} Medbring jeres svar fra opgaverne i Modul 03 til viderebearbejdning i klassen. \par Samarbejde diskuteres i mindre grupper og efterfølgende plenumdiskussion.

  \descitem{Skema for vejledningsmøder} Indkaldelse, dagsorden, referat, gruppelog.

\end{enumerate} & Sep 13 \\
  \hline
  37 & ProOnline & \begin{itemize}[noitemsep,leftmargin=*,topsep=0pt,partopsep=0pt]

  \item 04 Problemorienteret projektarbejde

  \item 05 Problemanalyse og problemformulering

\end{itemize} & Sep 16 \\
  \hline
  38 & Projekt & \textbf{Vejledning i projektgrupperne:} Samarbejdsaftale og vejlederaftale præsenteres for og godkendes af projektvejleder.

\par

Drøftelse af arbejdet i problemanalysefasen  på grundlag af Pro Online 04 og 05 & Sep 20 \\
  \hline
  38 & ProOnline & \begin{itemize}[noitemsep,leftmargin=*,topsep=0pt,partopsep=0pt]

  \item 06 Projektets faglige vidensgrundlag

  \item 07 Metoder og planlægning i projektarbejdet

\end{itemize} & Sep 23 \\
  \hline
\end{longtable}


\underline{Bemærk at problemanalysefasen starter i uge 39 samtidigt med at projektstarten slutter.}

\section*{Materialer}

Materialer der er særligt vigtige i projektstarten\footnote{Materialerne ligger på itslearning under \textsl{"General Course/Semester Information"}, \textsl{"Semesterprojekt og Projektbeskrivelse"} og \textsl{"Projektafleveringfasen og Evaluationfasen"}.}:
\begin{itemize}
  \item Semesterhåndbog
  \item Projektbeskrivelse
  \item Semesterplan
  \item Projektaflevering (inkl. rapportkontrolskema)
\end{itemize}

\section*{Skabelon til Gruppelog}
\label{sec:gruppelog}

\textbf{Uge xx}

Tirsdag - Projektarbejde
\begin{itemize}
  \descitem{Sted:}
  \descitem{Tidspunkt:}
  \descitem{Forberedelse:}
  \descitem{Dagsorden:}
  \descitem{Tilstede:}
  \descitem{Referat:}
\end{itemize}

Tirsdag - Vejledermøde
\begin{itemize}
  \descitem{Sted:}
  \descitem{Tidspunkt:}
  \descitem{Forberedelse:}
  \descitem{Dagsorden:}
  \descitem{Tilstede:}
  \descitem{Referat:}
\end{itemize}

Xxxdag - Projektarbejde
\begin{itemize}
  \descitem{Sted:}
  \descitem{Tidspunkt:}
  \descitem{Forberedelse:}
  \descitem{Dagsorden:}
  \descitem{Tilstede:}
  \descitem{Referat:}
\end{itemize}

\end{document}
