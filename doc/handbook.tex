\documentclass[a4paper, oneside]{memoir}

\usepackage[danish]{babel} % load typographical rules for the english language
\usepackage{graphics} % for \scalebox
\usepackage{hyperref} % for \href
\usepackage{xcolor} % for text color
\usepackage{enumitem} % for ordered and unordered list
\usepackage{graphicx} % for images
\usepackage{pdfpages} % for including pdfs
\usepackage{footnote} % for footnotes
\usepackage{longtable} % for tabular environment that spans multiple pages and supports footnotes
\usepackage{colortbl} % for cell coloring
\usepackage{multirow} % for \multicolumn

% https://github.com/latex3/babel/issues/51
\makeatletter\AtBeginDocument{\let\@elt\relax}\makeatother

% styling
\setsecnumdepth{subsubsection} % how deep to number sections
\setlength{\parindent}{0em} % horizontal indent for first line of paragraph
\setlength{\parskip}{1em} % vertical space between paragraphs

\newcommand{\textdesc}[1]{\textit{\textbf{#1}}}
\newcommand{\descitem}[1]{\item \textdesc{#1}}

\title{\documenttitle\\\scalebox{0.85}{\documentsubtitle}}
\author{Aslak Johansen \href{mailto:asjo@mmmi.sdu.dk}{asjo@mmmi.sdu.dk}\\Aisha Umair \href{mailto:aiu@mmmi.sdu.dk}{aiu@mmmi.sdu.dk}}

\begin{document}

\maketitle
\setcounter{tocdepth}{2}
\tableofcontentswrapper


%%%%%%%%%%%%%%%%%%%%%%%%%%%%%%%%%%%%%%%%%%%%%%%%%%%%%%%%%%%%%%%%%%%%%%%%%%%%%%%%
%%%%%%%%%%%%%%%%%%%%%%%%%%%%%%%%%%%%%%%%%%%%%%%%%%%%%%%%%%%%%%%%%%%%%%%%%%% Tema
\section*{Tema}

De første fire semestre på uddannelsen er sammenhængende tematiserede semestre.
Temaet for 1. semester og for semesterprojektet er \textbf{”Udvikling af softwareprogrammer”}.

%%%%%%%%%%%%%%%%%%%%%%%%%%%%%%%%%%%%%%%%%%%%%%%%%%%%%%%%%%%%%%%%%%%%%%%%%%%%%%%%
%%%%%%%%%%%%%%%%%%%%%%%%%%%%%%%%%%%%%%%%%%%%%%%%%%%%%%%%%%%%%%%%%%%%%%%% Indhold
\section*{Indhold}

Semestret har fire moduler, heraf tre fag og et semesterprojekt:
\begin{itemize}
  \item Fag
    \begin{itemize}
      \item Objektorienteret programmering 
      \item Computersystemer (Software engineering) eller
Introduktion til Cyberphysical Systemer (Softwareteknologi) 
      \item Statistisk Dataanalyse 
    \end{itemize}
  \item Semesterprojektet "Udvikling af softwareprogrammer", herunder:
    \begin{itemize}
      \item Online kursus i problemorienteret projektarbejde
      \item SDG kursus
    \end{itemize}
\end{itemize}

Yderligere detaljer kan findes her:
\begin{itemize}
  \item Software Engineering:
    \begin{itemize}
      \item Studieordning: \url{https://www.sdu.dk/da/om_sdu/fakulteterne/teknik/ledelse_administration/administration/studieordninger_a/software_civbach}
      \item Fagudbud 2022: \url{https://www.sdu.dk/da/om_sdu/fakulteterne/teknik/ledelse_administration/administration/studieordninger_a/software_civbach/moduler_e22}
    \end{itemize}
  \item Softwareteknologi:
    \begin{itemize}
      \item Studieordning: \url{https://www.sdu.dk/da/om_sdu/fakulteterne/teknik/ledelse_administration/administration/studieordninger_a/software_dipling}
      \item Fagudbud 2022: \url{https://www.sdu.dk/da/om_sdu/fakulteterne/teknik/ledelse_administration/administration/studieordninger_a/software_dipling/moduler_e22}
    \end{itemize}
\end{itemize}

%%%%%%%%%%%%%%%%%%%%%%%%%%%%%%%%%%%%%%%%%%%%%%%%%%%%%%%%%%%%%%%%%%%%%%%%%%%%%%%%
%%%%%%%%%%%%%%%%%%%%%%%%%%%%%%%%%%%%%%%%%%%%%%%%%%%%%%%%%%%%%%%%%%% Semesterteam
\section*{Semesterteam}

Semestret har et semesterteam bestående af semesterkoordinator, undervisere og projektvejledere.  Semesterkoordinator har det overordnede ansvar for semestret og semesterprojektet. Underviserne står for undervisningen i fagene, og underviserne har ansvaret for det faglige indhold af projektet. Tilrettelæggelsen af projektrammerne og vejledningen er hele teamets ansvar. Vejlederne står for vejledningen.

%%%%%%%%%%%%%%%%%%%%%%%%%%%%%%%%%%%%%%%%%%%%%%%%%%%%%%%%%%%%%%%%%%%%%%%%%%%%%%%%
%%%%%%%%%%%%%%%%%%%%%%%%%%%%%%%%%%%%%%%%%%%%%%%%%% Opdeling i Klasser og Grupper
\section*{Opdeling i Klasser og Grupper}

Der er på 1. semester 5 klasser med ca. 30 studerende i hver klasse. Hver klasse har en projektvejleder (i enkelte tilfælde to vejledere) og en instruktor for hvert af undervisningsfagene. Hver klasse består af ca. 5 projektgrupper. Projektgrupperne dannes 2 uger inde i semestret.

%%%%%%%%%%%%%%%%%%%%%%%%%%%%%%%%%%%%%%%%%%%%%%%%%%%%%%%%%%%%%%%%%%%%%%%%%%%%%%%%
%%%%%%%%%%%%%%%%%%%%%%%%%%%%%%%%%%%%%%%%%%%%%%%%%%%%%%%%%%%%%% Undervisningsform
\section*{Undervisningsform}

På ingeniøruddannelserne på det tekniske fakultet benyttes undervisningsmodellen Den syddanske Model for Ingeniøruddannelserne (DSMI).  I DSMl er der fokus på både de faglige kompetencer og de brede ingeniørkompetencer som samarbejdsevner, tolerance for andre, projektstyring, initiativ, ideskabelse/kreativitet og det at kunne tilegne sig viden. 

Undervisningen sker normalt i 4 timers blokke, hvor der er teori og praksis i samme undervisningsblok. 4-timers blokkene tilrettelægges forskelligt, men de har som regel to timers forelæsning og to timers øvelser og  tilrettelægges typisk ved en eller form for ”Flipped Class Room”, som indebærer at en stor del af studiearbejdet foregår som forberedelse til undervisningen. I forelæsningerne deltager flere eller alle klasser, mens øvelserne foregår i de enkelte klasser. 

Semesterprojektet starter fra begyndelsen af semestret og udføres parallelt med undervisningen. Projektarbejdet udføres i projektgrupper, hver gruppe har en vejleder. Noget af vejledningen gennemføres som projektseminarer i klasserne, noget som projekvejledning i projektgrupperne.  

%%%%%%%%%%%%%%%%%%%%%%%%%%%%%%%%%%%%%%%%%%%%%%%%%%%%%%%%%%%%%%%%%%%%%%%%%%%%%%%%
%%%%%%%%%%%%%%%%%%%%%%%%%%%%%%%%%%%%%%%%%%%%%%%%%%% Evaluering af Undervisningen
\section*{Evaluering af Undervisningen}

I løbet af semestret gennemføres to evalueringer: Midtvejsevaluering og slutevaluering. Midtvejsevalueringen har til formål at korrigere de studerendes indsats og forventninger og om nødvendigt at korrigere den konkrete undervisning og projektforløbet. Slutevalueringen har til formål at vurdere semestrets samlede indhold og forløb og give input til forbedringer.

%%%%%%%%%%%%%%%%%%%%%%%%%%%%%%%%%%%%%%%%%%%%%%%%%%%%%%%%%%%%%%%%%%%%%%%%%%%%%%%%
%%%%%%%%%%%%%%%%%%%%%%%%%%%%%%%%%%%%%%%%%%%%%% Semestermøder og Projektseminarer
\section*{Semestermøder og Projektseminarer}

Der holdes semestermøder og seminarer i løbet af semestret:
\begin{itemize}
  \item Semestermøder: Alle studerende og semesterkoordinator. Vejledere og undervisere kan deltage.
  \item Holdets time: Alle studerende og mentorer.
  \item Projektseminarer: Afholdes af vejleder for alle studerende i en klasse. 
  \item Møder med grupperepræsentanter: Semesterkoordinator og grupperepræsentanter. Møderne kan fx bruges til opfølgning på semesterprojektet eller til evalueringer. Vejledere og undervisere kan deltage.
  \item Møder i semesterteamet: Semesterkoordinator med undervisere og projektvejledere
\end{itemize}

Semestermøder, holdets time og projektseminarer fremgår af skemaet. Semesterkoordinatoren indkalder til semestermøder. Hvis der ønskes bestemte punkter på dagsordenen, gives der besked om det til semesterkoordinatoren to hverdage før mødernes afholdelse.

%%%%%%%%%%%%%%%%%%%%%%%%%%%%%%%%%%%%%%%%%%%%%%%%%%%%%%%%%%%%%%%%%%%%%%%%%%%%%%%%
%%%%%%%%%%%%%%%%%%%%%%%%%%%%%%%%%%%%%%%%%%%%%%%%%%%%%%%%%% E-learn - Itslearning
\section*{E-learn - Itslearning}

I modtager alle beskeder og informationer om og materialer til undervisning og projekt via ItsLearning. Beskeder gives hovedsageligt som annonceringer under \textbf{Oversigt}, som altid kan genfindes på ItsLearning.

For at lette brugen af ItsLearning bestræber underviserne sig på at bruge ItsLearning ensartet og i overensstemmelse med universitetets retningslinjer på tværs af kurser og semestre.

Pointgivende aktiviteter og opgaveafleveringer i alle kurser foregår også via ItsLearning og bedømmelser modtages via \textbf{Status og Opfølgning}.

Der findes en smartphone-applikation til ItsLearning, som er et godt og nemt supplement til browserudgaven.

%%%%%%%%%%%%%%%%%%%%%%%%%%%%%%%%%%%%%%%%%%%%%%%%%%%%%%%%%%%%%%%%%%%%%%%%%%%%%%%%
%%%%%%%%%%%%%%%%%%%%%%%%%%%%%%%%%%%%%%%%%%%%%%%%%%%%%%%%%%%%%%%%%%% Semesterplan
\section*{Semesterplan}

Semestret planlægges som en helhed og aktiviteterne på semestret koordineres af semesterteamet. Semesterplanen\footnote{Under planen "General Course/Semester Information" på itslearning.} giver en oversigt over vigtige datoer og frister herunder for tællende aktiviteter og afleveringer.

%%%%%%%%%%%%%%%%%%%%%%%%%%%%%%%%%%%%%%%%%%%%%%%%%%%%%%%%%%%%%%%%%%%%%%%%%%%%%%%%
%%%%%%%%%%%%%%%%%%%%%%%%%%%%%%%%%%%%%%%%%%%%%%%%%%%%%%%%%%%%%%%%%%%%%%%%% MitSDU
\section*{MitSDU}

Klik ind på \url{https://mitsdu.dk/da} for at se de oplysninger, der er relevant for dig som studerende ved SDU.

For oplysninger om studieordning, skemær, eksamen osv.:
\begin{itemize}
  \item Software Engineering: \url{https://mitsdu.dk/da/mit_studie/bachelor/softwareengineering_bachelor}
  \item Softwareteknologi: \url{https://mitsdu.dk/da/mit_studie/diplomingenioer/diplomingenioer_i_softwareteknologi}
\end{itemize}

\end{document}
